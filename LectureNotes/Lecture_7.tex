%%%%%%%%%%%%%%%%%%%%%%%%%%%%%%%%%%%%%%%%%
% Beamer Presentation
% LaTeX Template
% Version 1.0 (10/11/12)
%
% This template has been downloaded from:
% http://www.LaTeXTemplates.com
%
% License:
% CC BY-NC-SA 3.0 (http://creativecommons.org/licenses/by-nc-sa/3.0/)
%
%%%%%%%%%%%%%%%%%%%%%%%%%%%%%%%%%%%%%%%%%

%----------------------------------------------------------------------------------------
%	PACKAGES AND THEMES
%----------------------------------------------------------------------------------------

\documentclass{beamer}

\mode<presentation> {

% The Beamer class comes with a number of default slide themes
% which change the colors and layouts of slides. Below this is a list
% of all the themes, uncomment each in turn to see what they look like.

%\usetheme{default}
%\usetheme{AnnArbor}
%\usetheme{Antibes}
%\usetheme{Bergen}
%\usetheme{Berkeley}
%\usetheme{Berlin}
%\usetheme{Boadilla}
%\usetheme{CambridgeUS}
%\usetheme{Copenhagen}
%\usetheme{Darmstadt}
\usetheme{Dresden}
%\usetheme{Frankfurt}
%\usetheme{Goettingen}
%\usetheme{Hannover}
%\usetheme{Ilmenau}
%\usetheme{JuanLesPins}
%\usetheme{Luebeck}
%\usetheme{Madrid}
%\usetheme{Malmoe}
%\usetheme{Marburg}
%\usetheme{Montpellier}
%\usetheme{PaloAlto}
%\usetheme{Pittsburgh}
%\usetheme{Rochester}
%\usetheme{Singapore}
%\usetheme{Szeged}
%\usetheme{Warsaw}

% As well as themes, the Beamer class has a number of color themes
% for any slide theme. Uncomment each of these in turn to see how it
% changes the colors of your current slide theme.

%\usecolortheme{albatross}
%\usecolortheme{beaver}
%\usecolortheme{beetle}
%\usecolortheme{crane}
%\usecolortheme{dolphin}
%\usecolortheme{dove}
%\usecolortheme{fly}
%\usecolortheme{lily}
%\usecolortheme{orchid}
%\usecolortheme{rose}
%\usecolortheme{seagull}
%\usecolortheme{seahorse}
%\usecolortheme{whale}
%\usecolortheme{wolverine}

%\setbeamertemplate{footline} % To remove the footer line in all slides uncomment this line
%\setbeamertemplate{footline}[page number] % To replace the footer line in all slides with a simple slide count uncomment this line

%\setbeamertemplate{navigation symbols}{} % To remove the navigation symbols from the bottom of all slides uncomment this line
}

\usepackage{graphicx} % Allows including images
\usepackage{booktabs} % Allows the use of \toprule, \midrule and \bottomrule in 
%tables
\usepackage{listings}
\usepackage{hyperref}
\usepackage{subfig}
\usepackage[export]{adjustbox}
\usepackage{wrapfig}

%----------------------------------------------------------------------------------------
%	TITLE PAGE
%----------------------------------------------------------------------------------------

\title[Lecture 7]{Advanced R Programming - Lecture 7} % The short title 
%appears at the bottom of every slide, the full title is only on the title page

\author{Leif Jonsson} % Your name
\institute[STIMA LiU] % Your institution as it will appear on the bottom of 
%every 
%slide, may be shorthand to save space
{
Link\"{o}ping University \\ % Your institution for the title page
\medskip
\textit{leif.jonsson@ericsson.com\\leif.r.jonsson@liu.se} % Your email address
}
\date{\today} % Date, can be changed to a custom date

\addtobeamertemplate{navigation symbols}{}{ \hspace{1em}    
\usebeamerfont{footline}%
	\insertframenumber / \inserttotalframenumber }

\begin{document}

\begin{frame}
\titlepage % Print the title page as the first slide
\end{frame}

\begin{frame}
\frametitle{Today} % Table of contents slide, comment this block out to remove 
%it
\tableofcontents % Throughout your presentation, if you choose to use \section{} and \subsection{} commands, these will automatically be printed on this slide as an overview of your presentation
\end{frame}

%----------------------------------------------------------------------------------------
%	PRESENTATION SLIDES
%----------------------------------------------------------------------------------------

\begin{frame}
	\Huge{\centerline{Questions since last time?}}
\end{frame}

%------------------------------------------------
\section{Machine Learning}
%------------------------------------------------

\begin{frame}
	\frametitle{Machine learning?}
	\begin{center}
	Automatically detect patterns in data
	\end{center}
\end{frame}

\begin{frame}
	\frametitle{Machine learning?}
	\begin{center}
		Automatically detect patterns in data\\~\\
		Predict future observation
	\end{center}
\end{frame}

\begin{frame}
	\frametitle{Machine learning?}
	\begin{center}
		Automatically detect patterns in data\\~\\
		Predict future observation\\~\\
		Decision making under uncertainty
	\end{center}
\end{frame}

\begin{frame}
	\frametitle{Types of Machine learning}
	\begin{center}
		Supervised learning
	\end{center}
\end{frame}

\begin{frame}
	\frametitle{Types of Machine learning}
	\begin{center}
		Supervised learning\\~\\
		Unsupervised learning
	\end{center}
\end{frame}

\begin{frame}
	\frametitle{Types of Machine learning}
	\begin{center}
		Supervised learning\\~\\
		Unsupervised learning\\~\\
		Reinforcement learning
	\end{center}
\end{frame}

\begin{frame}
	\frametitle{Supervised learning}
	\begin{center}
		(also called predictive learning)\\~\\
		response variable\\~\\
		covariates/features\\~\\
		training set\\~\\
		$D = (x_i,y_i)_{(i=1)}^N$
	\end{center}
\end{frame}

\begin{frame}
	\frametitle{Supervised learning types}
	\begin{center}
		If $y_i$ is categorical:\\
		classification\\~\\
		If $y_i$ is real:\\
		regression
	\end{center}
\end{frame}

\begin{frame}
	\frametitle{Unsupervised learning}
	\begin{center}
		(also called knowledge discovery)\\~\\
		dimensionality reduction\\~\\
		latent variable modeling\\~\\
		$D = (x_i)_{(i=1)}^N$ \\~\\
		clustering, PCA, discovering of graph structures\\~\\
		data visualization\\~\\
	\end{center}
\end{frame}

\begin{frame}
	\frametitle{Curse of dimensionality}
	\begin{center}
		The more variables the larger distance between datapoints\\~\\
	\end{center}
\end{frame}

\begin{frame}
	\frametitle{Curse of dimensionality}
	\begin{center}
		The more variables the larger distance between datapoints\\~\\
		\begin{figure}
			\includegraphics[scale=0.2,valign=m]{figures/Curse-of-dimensionality}
			\label{fig:dim-curse}
		\end{figure}
		\href{http://www.newsnshit.com/curse-of-dimensionality-interactive-demo/}
		{source}
	\end{center}
\end{frame}

\begin{frame}
	\frametitle{Bias and variance in ML}
	\begin{center}
		\begin{figure}
			\includegraphics[scale=0.3,valign=m]{figures/data-bias-variance}
			\label{fig:bias-var}
		\end{figure}
		\medskip
		Underfit = high bias, low variance\\
		Overfit = low bias, high variance
	\end{center}
\end{frame}

\begin{frame}
	\frametitle{Model selection}
	\begin{center}
		bias and variance - tradeoff\\
	\end{center}
\end{frame}

\begin{frame}
	\frametitle{Model selection}
	\begin{center}
		bias and variance - tradeoff\\~\\
		hyper parameters
	\end{center}
\end{frame}

\begin{frame}
	\frametitle{Model selection}
	\begin{center}
		bias and variance - tradeoff\\~\\
		hyper parameters\\~\\
		generalization error
	\end{center}
\end{frame}

\begin{frame}
	\frametitle{Model selection}
	\begin{center}
		bias and variance - tradeoff\\~\\
		hyper parameters\\~\\
		generalization error\\~\\
		validation set/cross validation
	\end{center}
\end{frame}

\begin{frame}
	\frametitle{Predictive modeling pipeline}
	\begin{center}
		\begin{enumerate}
			\item Set aside data for test (estimate generalization error)
			\item Set aside data for validation (if hyperparams)
			\item Run algorithms
			\item Find best/optimal hyperparameters (on validation set)
			\item Choose final model
			\item Estimate generalization error on test set
		\end{enumerate}
	\end{center}
\end{frame}

\begin{frame}
	\frametitle{No free lunch theorem}
	\begin{center}
		different models work in different domains
	\end{center}
\end{frame}

\begin{frame}
	\frametitle{No free lunch theorem}
	\begin{center}
		different models work in different domains\\~\\
		accuracy-complexity-intepratability tradeoff
	\end{center}
\end{frame}

\begin{frame}
	\frametitle{No free lunch theorem}
	\begin{center}
		different models work in different domains\\~\\
		accuracy-complexity-intepratability tradeoff
	\end{center}
\end{frame}

\begin{frame}
	\frametitle{No free lunch theorem}
	\begin{center}
		different models work in different domains\\~\\
		accuracy-complexity-intepratability tradeoff\\~\\
		...but more data always wins
	\end{center}
\end{frame}

%------------------------------------------------
\section{Supervised learning in R}
%------------------------------------------------

\begin{frame}
	\frametitle{the caret package}
	\begin{center}
		package for supervised learning
	\end{center}
\end{frame}

\begin{frame}
	\frametitle{the caret package}
	\begin{center}
		package for supervised learning\\~\\
		does not contain methods - a framework
	\end{center}
\end{frame}

\begin{frame}
	\frametitle{the caret package}
	\begin{center}
		package for supervised learning\\~\\
		does not contain methods - a framework\\~\\
		compare methods on hold-out-data
	\end{center}
\end{frame}

\begin{frame}
	\frametitle{the caret package}
	\begin{center}
		package for supervised learning\\~\\
		does not contain methods - a framework\\~\\
		compare methods on hold-out-data\\~\\
		\href{http://topepo.github.io/caret/}{http://topepo.github.io/caret/}
	\end{center}
\end{frame}

\begin{frame}
	\frametitle{the caret package}
	\begin{center}
		package for supervised learning\\~\\
		does not contain methods - a framework\\~\\
		compare methods on hold-out-data\\~\\
		\href{http://topepo.github.io/caret/}{http://topepo.github.io/caret/}\\~\\
		specific algorithms are part of other courses
	\end{center}
\end{frame}

%------------------------------------------------
\section{Probability in R}
%------------------------------------------------

\begin{frame}
	\frametitle{the caret package}
	\begin{center}
	\begin{table}
		\begin{tabular}{l | l | l}
			\textbf{Prefix} & \textbf{Description} & \textbf{Example}\\
			\hline
			r & Random draw & rnorm \\
			\hline
			d & Density function & dbinom \\
			\hline
			q & Quantile function & qbeta \\
			\hline
			p & CDF & pgamma \\
		\end{tabular}
	\end{table}
\end{center}
\end{frame}

%------------------------------------------------
\section{Big data}
%------------------------------------------------

\begin{frame}
	\frametitle{Big data}
	Big data is like teenage sex:
	everyone talks about it,
	nobody really knows how to do it,
	everyone thinks everyone else is doing it,
	so everyone claims they are doing it...\\
	- Dan Ariely
\end{frame}

\begin{frame}
	\frametitle{Big data is relative...}
	\begin{center}
		... to computational complexity\\~\\
		$O(N):10^{12}$
	\end{center}
\end{frame}

\begin{frame}
	\frametitle{Big data is relative...}
	\begin{center}
		... to computational complexity\\~\\
		$O(N):10^{12}$\\
		$O(N^2):10^{6}$
	\end{center}
\end{frame}

\begin{frame}
	\frametitle{Big data is relative...}
	\begin{center}
		... to computational complexity\\~\\
		$O(N):10^{12}$\\
		$O(N^2):10^{6}$\\
		$O(N^3):10^{4}$\\
	\end{center}
\end{frame}

\begin{frame}
	\frametitle{Big data is relative...}
	\begin{center}
		... to computational complexity\\~\\
		$O(N) : 10^{12}$\\
		$O(N^2) : 10^{6}$\\
		$O(N^3) : 10^{4}$\\
		$O(2^N) : 50$\\
	\end{center}
\end{frame}

\begin{frame}
	\frametitle{Big data is relative...}
	\begin{center}
		... to computational complexity\\~\\
		$O(N) : 10^{12}$\\
		$O(N^2) : 10^{6}$\\
		$O(N^3) : 10^{4}$\\
		$O(2^N) : 50$\\~\\
		We need algorithms that scales!
	\end{center}
\end{frame}

\begin{frame}
	\frametitle{Big data is relative...}
	\begin{center}
		... to computational complexity\\~\\
		$O(P^2 * N) : \text{Linear regression}$
	\end{center}
\end{frame}

\begin{frame}
	\frametitle{Big data is relative...}
	\begin{center}
		... to computational complexity\\~\\
		$O(P^2 * N) : \text{Linear regression}$\\
		$O(N^3) : \text{Gaussian processes}$
	\end{center}
\end{frame}

\begin{frame}
	\frametitle{Big data is relative...}
	\begin{center}
		... to computational complexity\\~\\
		$O(P^2 * N) : \text{Linear regression}$\\
		$O(N^3) : \text{Gaussian processes}$\\
		$O(N^2) / O(N^3) : \text{Support vector machines}$\\
	\end{center}
\end{frame}

\begin{frame}
	\frametitle{Big data is relative...}
	\begin{center}
		... to computational complexity\\~\\
		$O(P^2 * N) : \text{Linear regression}$\\
		$O(N^3) : \text{Gaussian processes}$\\
		$O(N^2) / O(N^3) : \text{Support vector machines}$\\
		$O(T(P * N * log(N)) : \text{Random forests}$\\
	\end{center}
\end{frame}

\begin{frame}
	\frametitle{Big data is relative...}
	\begin{center}
		... to computational complexity\\~\\
		$O(P^2 * N) : \text{Linear regression}$\\
		$O(N^3) : \text{Gaussian processes}$\\
		$O(N^2) / O(N^3) : \text{Support vector machines}$\\
		$O(T(P * N * log(N)) : \text{Random forests}$\\
		$O(I * N) : \text{Topic models}$\\
	\end{center}
\end{frame}

\begin{frame}
	\frametitle{Big data in R}
	\begin{center}
		R stores data in RAM\\~\\
	\end{center}
\end{frame}

\begin{frame}
	\frametitle{Big data in R}
	\begin{center}
		R stores data in RAM\\~\\
		integers\\
		4 bytes\\
		numerics\\
		8 bytes\\~\\
	\end{center}
\end{frame}

\begin{frame}
	\frametitle{Big data in R}
	\begin{center}
		R stores data in RAM\\~\\
		integers\\
		4 bytes\\
		numerics\\
		8 bytes\\~\\
		A matrix with 100m rows and 5 cols with numerics \\
		$100000000 * 5 * 8 / (1024^3) \approx 3.8$
	\end{center}
\end{frame}

\begin{frame}
	\frametitle{How to handle}
	\begin{center}
		Handle chunkwise\\
		Subsampling\\
		More hardware\\
		C++/Java backend (dplyr)\\
		Reduce data in memory\\
		Database backend\\
	\end{center}
\end{frame}

\begin{frame}
	\frametitle{If not enough}
	\begin{center}
		Spark and SparkR\\~\\
		Fast cluster computations  for ML /STATS\\~\\
		\href{https://www.youtube.com/watch?v=_Ss1Cm6WO-I}
		{introduction to Spark}	\\
	\end{center}
\end{frame}

%------------------------------------------------
\section{Data munging}
%------------------------------------------------

\begin{frame}
	\frametitle{Tidy data}
	\begin{center}
		Theoretical approach to data handling\\~\\
		\textbf{Tidy} data and \textbf{messy} data\\~\\
	\end{center}
\end{frame}

\begin{frame}
	\frametitle{Tidy data}
	\begin{center}
		\begin{enumerate}
			\item Each variable forms a column
			\item Each observation forms a row
			\item Each type of observational unit forms a table
		\end{enumerate}
	\end{center}
\end{frame}

\begin{frame}
	\frametitle{Tidy data}
	\begin{center}
		\begin{enumerate}
			\item Each variable forms a column
			\item Each observation forms a row
			\item Each type of observational unit forms a table
		\end{enumerate}
		\medskip
		Examples: \texttt{iris} and \texttt{faithful}
	\end{center}
\end{frame}


\begin{frame}
	\frametitle{Why tidy?}
	\begin{center}
		80 \% of Big Data work is data munging\\
	\end{center}
\end{frame}

\begin{frame}
	\frametitle{Why tidy?}
	\begin{center}
		80 \% of Big Data work is data munging\\~\\
		Analysis and visualization is based on tidy data
	\end{center}
\end{frame}

\begin{frame}
	\frametitle{Why tidy?}
	\begin{center}
		80 \% of Big Data work is data munging\\~\\
		Analysis and visualization is based on tidy data\\~\\
		Performant code
	\end{center}
\end{frame}

\begin{frame}
	\frametitle{Why tidy?}
	\begin{center}
		80 \% of Big Data work is data munging\\~\\
		Analysis and visualization is based on tidy data\\~\\
		Performant code \\~\\
	\end{center}
\end{frame}

\begin{frame}
	\frametitle{Data analysis pipeline}
	\begin{center}
		Messy data $\rightarrow$ Tidy data $\rightarrow$ Analysis
	\end{center}
\end{frame}

\begin{frame}
	\frametitle{Messy data}
	\begin{center}
		\begin{enumerate}
			\item Column headers are values, not variable names. 
			(\texttt{AirPassengers})
		\end{enumerate}
	\end{center}
\end{frame}

\begin{frame}
	\frametitle{Messy data}
	\begin{center}
		\begin{enumerate}
			\item Column headers are values, not variable names. 
			(\texttt{AirPassengers})
			\item Multiple variables are stored in one column. (\texttt{mtcars})
		\end{enumerate}
	\end{center}
\end{frame}

\begin{frame}
	\frametitle{Messy data}
	\begin{center}
		\begin{enumerate}
			\item Column headers are values, not variable names. 
			(\texttt{AirPassengers})
			\item Multiple variables are stored in one column. (\texttt{mtcars})
			\item Variables are stored in both rows and columns. 
			(\texttt{crimetab})
		\end{enumerate}
	\end{center}
\end{frame}


\begin{frame}
	\frametitle{Messy data}
	\begin{center}
		\begin{enumerate}
			\item Column headers are values, not variable names. 
			(\texttt{AirPassengers})
			\item Multiple variables are stored in one column. (\texttt{mtcars})
			\item Variables are stored in both rows and columns. 
			(\texttt{crimetab})
			\item Multiple types of observational units are stored in the same 
			table.
		\end{enumerate}
	\end{center}
\end{frame}

\begin{frame}
	\frametitle{Messy data}
	\begin{center}
		\begin{enumerate}
			\item Column headers are values, not variable names. 
			(\texttt{AirPassengers})
			\item Multiple variables are stored in one column. (\texttt{mtcars})
			\item Variables are stored in both rows and columns. 
			(\texttt{crimetab})
			\item Multiple types of observational units are stored in the same 
			table.
			\item A single observational unit is stored in multiple tables.
		\end{enumerate}
	\end{center}
\end{frame}

\begin{frame}
	\frametitle{dplyr}
	\begin{center}
		\begin{enumerate}
			Verbs for handling data\\~\\
			Highly optimized C++ code (backend)\\~\\
			Handling larger datasets in R\\
			(no copy-on-modify)
		\end{enumerate}
	\end{center}
\end{frame}

\begin{frame}
	\frametitle{dplyr+tidyr}
	\begin{center}
		\href{https://www.rstudio.com/wp-content/uploads/2015/02/data-wrangling-cheatsheet.pdf}
			{the cheatsheet}
	\end{center}
\end{frame}

%------------------- THE END ---------------------------------------------------

\begin{frame}
\Huge{\centerline{The End... for today.}}
\Huge{\centerline{Questions?}}
\Huge{\centerline{See you next time!}}
\end{frame}

%-------------------------------------------------------------------------------

\end{document} 